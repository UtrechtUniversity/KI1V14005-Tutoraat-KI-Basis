\chapter{Meeting Guidelines}

The following are non-binding guidelines, feel free to do this in any way you please as long as you reach the aims.

\section{Get-to-know Meeting}

\begin{itemize}

	\item \textbf{Form}: Group meeting.

	\item \textbf{Aims}:

	\begin{itemize}

		\item tutor and students introduce themselves to each other

		\item information about the tutoring program is dispersed

		\item expectations for block 1 are discussed

	\end{itemize}

	\item \textbf{Preparation} (Students): None.

	\item \textbf{Preparation} (Tutor): Familiarize yourself with the program.

	\item \textbf{Proposed Schedule} (ca. 1h):

	\begin{enumerate}

		\item Tutor introduces themselves (background, teaching, research) and gives contact info (email, office hours if available).

		\item Tutor answers questions about program and gives schedule for future meetings.

		\item Students introduce themselves (e.g. say your name, your favorite X, and all the names and X's of people before you).

		\item Tips and tricks activity (slips (see appendix) are distributed to the students, each student asks another student the question on their slip by name, answers are shared by group, tutor, and mentors)

		\item Open end, fade into mentor activities.

	\end{enumerate}

	\item \textbf{Remarks}:

	\begin{itemize}

        \item If you have more than one group, an ad hoc solution for the meetings is to let the mentors run them and go back-and-forth between the two groups.
          We'll think about a general solution.

        \item The above schedule is just a suggestion, there are other ways to conduct the meeting, it's up to you.
          E.g. some tutors have had the mentors explain the program (using an INCOGNITO issued guideline, I'll see if I can get it) first.

	\end{itemize}

\end{itemize}

\section{Break Day Meeting}

\begin{itemize}


	\item \textbf{Form}: Group meeting.

	\item \textbf{Aims}:

		\begin{itemize}

			\item reflection on experiences in block 1

			\item planning of further meetings

			\item re-establish contact

		\end{itemize}

	\item \textbf{Preparation} (Students): Answer the questionnaire and put it on BB.

	\item \textbf{Preparation} (Tutor): Have a look at the questionnaire. (But no need to read all the answers)

	\item \textbf{Proposed Schedule} (ca. 1h):

	\begin{itemize}

		\item Tutor opens the meeting: (re)-introduces themselves, explains the purpose of and plan for the meeting.

		\item Question activity: the students have prepared answers to a series of questions as the basis for the meeting:

		\begin{itemize}

			\item Try to get the students to talk about this.

			\item Start with the ``funniest moment'' to break the ice.

			\item Move to more difficult issues.

			\item Encourage the group to share.

			\item \textbf{Identify students to talk to individually.}

		\end{itemize}

		\item Tutor closes the meeting: mentions further meetings (progress + choice, possibly scheduling them), repeats how to reach them.


	\end{itemize}

	\item \textbf{Remarks}: If you have more than one group, an ad hoc solution for the meetings is to let the mentors run them and go back-and-forth between the two groups. We'll think about a general solution.

\end{itemize}


\section{Progress Meeting}

	\begin{itemize}

		\item \textbf{Form}: Individual meetings

		\item \textbf{Aims}:

		\begin{itemize}

			\item check in with the students

			\item identify students who are struggling and help find solutions

			\item identify students who want to make use of the February 1 rule for unenrollment

		\end{itemize}

		\item \textbf{Preparation} (Students): Fill in the reflection form, submit on BB, \textbf{deadline}: 21/12/2019

		\item \textbf{Preparation} (Tutor): Read the reflection forms filled in by the students.

		\item \textbf{Proposed Schedule} (ca. 10-15 minutes per meeting):

		\begin{itemize}

			\item Ask the student how they are doing (which courses did they pass/fail, is there something they're struggling with---offer pointers if needed)

			\item Ask the student if they know about the BSA and if they think they'll get it (offer info if needed, explain the Februrary 1 rule)

			\item Mention that there will be another meeting about choice of track.

			\item If you think a student is honors material, point them towards that (link in the appendix).

			\item Look at the curriculum (excel sheet on BB) together with the student, explain the courses that are coming up (upshot: Inleiding Logica en tot Cognitiewetenschappen are relatively easy, block 2 courses are hard but it get's (somewhat) easier again)

			\item Give room for questions (typical questions at this stage: minor, studying abroad, additional courses/honors, writing problems, psychological problems, problems with course registration---see appendix with info about this issues)

		\end{itemize}

		\item \textbf{Remarks}.

		\begin{itemize}

			\item It would be nice if you could keep a list (confidential) of who you think is at risk for failing their BSA. We'll do some statistics at the end of term.

			\item You don't need to give an advise yourself to the student (continue or discontinue), you help them make an informed decision themselves (they are adults).
                          Help them find the information they need (using the resources in the appendix).

			\item Doodle (\url{https://doodle.com}) can be used fruitfully for letting students sign-up for the meeting (create a doodle with the time-slots you have available, allow each participant to only select one option, make sure that participants can't see each others selections).

			\item If you need to book a room, contact the OW secretaries (see Appendix).

			\item \textbf{Afterwards, please create a note in Osiris docent (see Appendix) containing any important information that's been discussed in the meeting.}

		\end{itemize}

	\end{itemize}

\section{Choice Meeting}

[tbc]

%%% Local Variables:
%%% mode: latex
%%% TeX-master: "../../main"
%%% End:
