\chapter{Tipps and Tricks (Opleidingsintroductie)}


\begin{itemize}

	\item Je wilt weten welke vakken je allemaal kunt volgen (in een bepaalde periode), waar ga je zoeken?

	\item Je hebt moeite met een huiswerkopdracht, wat kan je doen?

	\item Je wilt graag oefenen voor een tentamen, waar vind je oefenmateriaal?

	\item Ik wil eigenlijk meer (of minder!) uitdaging; hoe krijg ik dit voor elkaar?

	\item Hoe haal je meer uit hoorcolleges? Hoe kun je ze nuttiger maken?

	\item Wat doe je als je samen met mede-studenten wil studeren of werken aan een opdracht.

	\item Wat is een goede locatie om rustig te kunnen studeren?

	\item Wat is een goede plek of gelegenheid om te socialisen met andere (KI) studenten?

	\item De docent gaat in een hoog tempo door de stof heen, zou je daar iets aan proberen te doen, zo ja wat?

	\item De werkdruk voor een vak is veel groter dan wat er voor staat en je andere vak lijdt eronder. Wat doe je?

	\item Wat doe je als de docent laat is met opdrachten en informatie aanleveren?

	\item Er zijn wat problemen in je persoonlijke situatie en je studie lijdt eronder, bij wie kan je hiervoor het beste terecht?

	\item Je denkt dat je in de eerste twee blokken een vak niet gaat halen. Wat doe je?

	\item Wat als je je al een tijd niet zo lekker in je vel voelt?

	\item Anders: namelijk....
(wat moeten anderen echt niet missen!?)

\end{itemize}


%%% Local Variables:
%%% mode: latex
%%% TeX-master: "../../main"
%%% End:
